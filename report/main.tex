\documentclass[letterpaper,twocolumn,10pt]{article}

\usepackage[margin=1in,columnsep=0.33in]{geometry} \usepackage{fancyhdr}
\usepackage{amsmath} \usepackage{hyperref}

\pagestyle{fancy} \fancyhf{} % clear header and footer \fancyfoot[C]{\thepage}
% add page number to center footer

\begin{document}

\title{CS438 Project Report}
\author{Aarya Bhatia}

\date{\today}
\maketitle
% \tableofcontents

\begin{abstract} In this project I implemented a client and server program
	following the IRC (Inter Relay Chat) protocol in the C language. The
	IRC is a distributed chat server protocol which enables message delivery
	between clients that are connected to different servers in real time. An
	IRC server does this by relaying the messages to its peers.

	The goal of this project is to get familar with distributed systems and
	learn about an important networking protocol used by many internet users.
	This project taught me the networking protocols used at a large scale and
	gave me insight about the kind of problems we can come across. This project
	also gave me experience in programming, debugging and testing distribtued
	systems.

	For this project, I followed the RFC 1459 given in
	\url{https://www.rfc-editor.org/rfc/rfc1459} for the IRC Server
	specifications.

	The source code for this project can be found on Github here:
	\url{https://github.com/aarya-bhatia/irc}. The repo contains a \verb|README.md|
	file that describes how to build and use the programs.

	Here is a demo video for this project on YouTube: \url{https://www.youtube.com/watch?v=MTd-5KoxiPo}

\end{abstract}

\section{Implementation}

The IRC is an application layer protocol used for real time text communication
between multiple users on the Internet. The protocol was developed in the late
1980s and has since become a popular means of online communication.

The server is implemented as a single-threaded event-driven application. It
uses the linux polling API and non-blocking TCP sockets for all network
communication. This allows the server to service many requests at once and be
more efficient. The server uses buffers and message queues for communication.
There are various data structures like the \verb|Connection struct| to handle
the communication for each connection. The message queue allows the server to
prepare messages to be sent to a client before the client is ready to receive
them. It also lets the server parse multiple messages sent by the client at the
same time and save them for future. The buffers store the current request or
response data. Since the connections are multiplexed, these buffers may be in
an incomplete state. However, when the messages are fully received or sent, the
message queues will always be in a correct state.

An IRC network is composed of many servers running in parallel as well as many
clients connected to a single server. The network topology is always a spanning
tree by design. This ensures that there are no loops in the network. Therefore,
servers can relay messages to every connected peer to propogate messages from
one server to the others. Secondly, there is always a single path between each
pair of clients due to the nature of the network. This eliminates the risk of
duplicate messages and infinite cycles of relayed messages.

The servers in the IRC network need to keep the state of the network in sync with each
other. A server keeps track of its own information as well as the information
of clients and peers connected to it. The server also keeps track of other
servers and clients that it can reach through its peers. With this
information, the server implements the cycle detection logic so that any
illegal connection can be dismissed immediately. The server also implements
nick collision detection and has the ability to check if a nick exists on
the network.

The client and server use a config file to configure the servers. Only servers
specified in the file are allowed to participate in the network. This is
recommended by the RFC. The config file contains the name of the server, the IP
address, port number and password as a CSV file. The password is used for
server-server registration. The port is where the server listens for new
connections from clients or peers. It is possible to edit this file to add or
remove a server. Here is an example of a valid entry in the \verb|config.csv| file:
\verb|testserver,192.168.64.4,5000,password@1234|

The client is implemented as a multi-threaded application so that it can
communicate with the user on stdin as well as the server over a socket. The
client also uses the \verb|Connection struct| and messages queues like the
server. The client message queues are synchronised by mutex locks so messages
can be sent from various threads. Most commands read from stdin are sent
verbatim to the server as plain-text, with only the addition of a CRLF
delimiter.

A user can start the client by typing \verb|./build/client <ServerName>| on the
terminal, where \verb|ServerName| is a server that exists in the config file.
The client will try to establish a TCP connection with the IP and port
as specified by the entry in the config file. The user can interact with
the client by typing IRC commands such as ``NICK example" or \verb|PRIVMSG \#aarya :Hello|.
The new line character is used to end the messages. All messages are at most 512 bytes.
The client also displays the replies sent by the server to stdout asynchronously.

The client can accept special commands that are prefixed with a `/'. These
commands allow us to perform special actions like registration. We can use the
client to behave as a user or another server (only for testing). For users, the
client can recognise a text file that contains the username, realname and nick
for that user. This allows the user to login with \verb|/client filename|. The
client will load all the user details from the specified file and make the
registration requests to the server on its own. These requests include the NICK
and USER message.

For example, a client can create a file `login.txt' with the following
contents: \verb|aarya aaryab2 :Aarya Bhatia|. Now the client can start the
program and type \verb|/client login.txt|. The client will use this file to
create the initial registration request and push them to the message queue,
ready to be sent to the server. The server registers the client if the request
succeeds and it notifies the other servers about the users existence. This
enables the new user to receive messages from other users on the network. The
registration can fail for various reasons, for example the NICK chosen by the
user may be in use. Furthermore, the requests sent by the client may be
malformed. In any case, the server will remove the connection with the client
and no other messages will be exchanged.

After registration, the user can quit the client program by typing the
\verb|QUIT [:<reason>]| command. This will gracefully stop the client. There
will be a final exchange of messages between the client and server to faciliate
the dismissal of the client from the network. The exact process for quitting is
explained in the `Client' section. When a user quits the network, the server must update the data structures on
its side, and notify the entire network that the client has left. The other
servers recursively update and notify their peers until everyone in the network knows
that the client has left.

For chatting between users or on a channel, we use the \verb|PRIVMSG| or \verb|NOTICE| command.
The difference is that the NOTICE command does not generate automatic replies even on an error.
The NOTICE command is primarily meant for bots but has the exact same format as PRIVMSG. These commands
are discussed later.

To use the client program in `server mode', we can run a similar command on the client such
as \verb|/register ServerName|. Here \verb|ServerName| refers to a server in the
config file as before. This command establishes a server-server connection where the client is acting as the 'other' server.
This mode is useful for testing.

The initial messages sent in a server-server connection are different from that of a
client-server connection. For a server-server connection, the IRC protocol
requires that a ``SERVER" and ``PASS" message is exchanged between both parties -
the receiver and the sender. The server initiating the connection will be
called the `ACTIVE SERVER' whereas the opposite server will be called the
`PASSIVE SERVER'. In a client-server connection we use the `NICK' and `USER' message.


\section{Server}

The server handles three kinds of polling events for each socket connection:

\begin{itemize}
	\item If the event occurs on the server's listening socket, it indicates
	      to the server that the
	      server can accept new connections.
	\item On the event of an error, a client has disconnected. A disconnection event
	      is of importance because the server should inform the rest of the network
	      about this client. The server also needs to update its hashtables to remove this
	      user wherever required. The server notifies the network by sending a
	      \verb|KILL <client>| message and fills in the nick of the disconnecting client.
	\item On a write event, we send any pending messages from that user's
	      message queue
	      if any. We can prepare messages to be sent to any client by adding them
	      to
	      their message queues. The messages will eventually get delivered to the
	      client
	      when they are ready to receive them.
	\item On a read event from a client, we receive any data they have to send
	      and parse the messages if it is completed. A message is `completed'
	      when the CRLF delimiter is found in the message. Moreover, every IRC
	      message has a maximum length of 512 bytes. If a message exceeds this
	      limit without a CRLF we return an error status and the server can
	      remove the client sending the malformed request. On a successful but
	      incomplete message, we store the bytes in the user's request buffer.
	      Messages are transferred to the queue only when they are completed. A
	      message that is malformed or contains an invalid command may generate
	      an error response or it may be ignored.
\end{itemize}

The main structs used by the Server are the \verb|Peer|, \verb|User|,
\verb|Connection| and \verb|Server| struct. The following sections discuss
some of these data types in more detail.

\section{Connection struct}

\begin{itemize}
	\item The connection struct is a generic type which helps us read or write
	      data to
	      any socket connection. It is used by both client and server.
	\item The connection struct has a type parameter which can either be
	      \verb|UNKNOWN_CONNECTION|, \verb|CLIENT_CONNECTION|, \verb|PEER_CONNECTION|, or
	      \verb|USER_CONNECTION|.
	\item A new connection on the server is set to be \verb|UNKNOWN|. It is
	      later
	      promoted to a \verb|PEER| or a \verb|USER| connection when the client
	      sends the
	      initial messages i.e. a user would send a \verb|NICK/USER| pair and a
	      server
	      would send a \verb|PASS/SERVER| to register themselves.
	\item Each connection can also store an arbitary data pointer. This
	      parameter is
	      used to store a pointer to a Peer or User struct depending on the type
	      of
	      connection. It is initialised when the connection type is determined.
	      The
	      request handlers only deal with this data interact with the Connection
	      through
	      the `message queue'.
	\item Message Queues and Message Buffers: A connection struct contains
	      incoming and
	      outgoing message queues in addition to request and response buffers.
	      This
	      allows us to send or receive multiple messages from a client at once.
	      It is
	      also used for storing chat messages from another client. Since we may
	      not send
	      or receive all the data at once (due to nonblocking IO), we also keep
	      track of
	      where we are in the buffers using integer indices and offsets. All of
	      this
	      logic is encapsulated by the Connection struct.
	\item Note that the User and Peer struct have their own internal message
	      queues.
	      Initially messages are put in the main message queue, but after client
	      has
	      registered, the messages are put in the internal message queues. This
	      is done
	      so we don't have to deal with Connection structs in the request handler
	      functions and we do not have to cast the data to the right type. We
	      also create
	      an abstraction between the client and the connection.
\end{itemize}

\section{Server struct}

The server data is stored in the Server struct. The server contains data about
all of its clients, users and the network - such as the nicks that exist in the
network. The servers must know this information to avoid nick conflicts. Also,
the server uses this data to route messages.

The Server uses the following data structures:

\begin{itemize}

	\item A hashtable \verb|connections| is used to map sockets to connection
	      structs
	      for each connection, i.e clients and peers.
	\item A hashtable \verb|nick_to_user_map| to map a nick string to a user
	      struct for
	      a user. The users get a random nick in the beginning. When the user
	      updates
	      their nick, the entry in the hashmap for that user is also updated.
	      This map is
	      useful to check which nicks are available and also access the user's
	      data when
	      we want to deliver a message to that user, but only knowing their nick.
	\item A hashtable \verb|name_to_channel_map| to map each channel name to a
	      channel
	      struct. Each server has their own copy of the channel. Since different
	      users
	      are connected to different servers, a channel message must be
	      propogated to the
	      entire network in order to reach all channel members. But a single
	      server does
	      not know all the clients in the channel.
	\item A hashtable \verb|name_to_peer_map| is used to map the name of a peer
	      server
	      to a Peer struct. When a server-to-server connection is established the
	      remote
	      server becomes a peer for the current server. The server which
	      initiates the
	      connection is known as the \verb|ACTIVE_SERVER| and the server which
	      accepts the
	      connections is known as \verb|PASSIVE_SERVER|. The entry is added after
	      registration
	      as the peer name is not known before the SERVER message is received
	      The IRC does not enforce `default names' unlike the case with users.
	\item The entry for this peer is removed when the peer quits or disconnects.
	      The server that
	      discovered
	      the disconnected peer must generate a \verb|SQUIT| message to notify
	      the other
	      servers in the network as well. This has a similar purpose as the
	      \verb|KILL|
	      message for user disconnections.
	\item A hashtable \verb|nick_to_serv_name_map| is used to determine which
	      users are
	      connected to each server. This map is updated when a peer advertises a
	      new user
	      connection or relays the message from another server. This map helps
	      keep track
	      of all the users on the network at each server. The entries are removed
	      when a
	      user disconnects and a server sends a KILL request for that user.

\end{itemize}

\section{Parser}

I implemented a simple message parser which is used to parse messages and check
for errors. This is used by both server and client. The details about the
syntax of the message is given in the RFC and is out of scope of this report.
At a simple level, an IRC message includes the following tokens in the message:

\begin{itemize}

	\item source: This identifies the sender of the message. Clients never fill
	      in this
	      field. However, a server fills in this field on behalf of the client
	      for
	      commands like \verb|PRIVMSG|. The server always fills in this field
	      with their
	      own server name when sending a message to a user or peer. When this
	      source
	      field is used, it begins with a `:' character.
	\item command: This field is filled by a `command' if the message is a
	      request from
	      the client. It is filled by the `reply' if the message is a response
	      from the
	      server. For example, when a user sends a \verb|PRIVMSG|, then
	      \verb|PRIVMSG| is
	      the command. When the server sends a message, often times the `command'
	      is
	      filled by a numerical code. A list of all numerical replies can be
	      found in the
	      RFC and also in the \verb|replies.h| file included in the source code.
	\item parameters: There are 15 parameters that we can use in the IRC
	      protocol.
	      These are filled with any parameters we might want to send with a
	      message.
	      Usually, the first parameter is the `target'. For example, in a
	      \verb|PRIVMSG|,
	      this target can be the nick of the receipeint of the message.
	      Parameters do not
	      contain any spaces because space is the separator used between
	      parameters.
	\item body: This is considered as the final parameter and can contain
	      arbitary
	      text. This is where the user would add their `message' in a
	      \verb|PRIVMSG|
	      command. The body is prefixed by another `:' character, and can contain
	      spaces.
	      This field is also used for the `realname' of the user in a `USER'
	      message,
	      because the `realname' can contain spaces. The servers often use this
	      field to
	      describe the reply, as most IRC clients hide the `parameters' from the
	      user.
	      They usually use this field to display to the user.

\end{itemize}

\section{Commands}

The server currently supports the following commands from the clients: MOTD,
NICK, USER, PING, QUIT, HELP, PRIVMSG, NOTICE, INFO, WHO, NAMES, LIST, PART,
JOIN, TOPIC, CONNECT. There is a special command TEST\_LIST\_SERVER which is
discussed later.

\subsection{QUIT}

\begin{itemize}

	\item The QUIT command is used by a client to indicate their wish to leave
	      the server.
	\item All data associated with this user is freed and socket is closed.
	\item An ERROR reply is sent before closing the socket to allow the reader
	      thread in the client to quit gracefully.
	\item A client is not removed immediately. Instead we set a \verb|quit|
	      flag in the user data because we still need to send the final messages.

	\item When all messages are sent and the user is marked as quit, the server
	      closes the connection.

\end{itemize}

\subsection{MOTD}

\begin{itemize}

	\item This command sends the "Message of the Day" to the user. - It looks
	      up the current message from a file (motd.txt) that contains a list of
	      quotes.
	\item This file was downloaded using the \verb|zenquotes.io| API using a
	      small python script in \verb|download_quotes.py|.
	\item This list can be updated by running this script at any point.
	\item The server fetches the line that is at position equal to
	      \verb|day_of_year % total_lines|, where \verb|total_lines < 365|.
	\item The filename can be changed at any time as the server reopens the
	      file each time. It is not neccessary to cache this file as the MOTD is
	      only sent once for each client on registration or if someone requests
	      for it.

\end{itemize}

\subsection{NICK/USER}

These two commands are used for user registration.

\begin{itemize}

	\item A user can register with NICK and USER commands - NICK can be used to
	      update nick at any time

	\item The username and realname set by the USER command cannot change.

	\item Users can use NICKs to send messages to another user. (See PRIVMSG
	      and NOTICE for more information).

	\item A user's nick is freed i.e. made available to other users when they
	      leave the session.

	\item A server always knows every client on the network.

	\item NOTE: There is a particular scenario known as \textbf{net split} where
	      we experience a NICK collision and we cannot guard against it:
	      Suppose there are two \textit{disjoint} IRC networks
	      containing a user with the same NICK. Suppose that these two networks are joined by a
	      server-server connection. This results in us having two clients with same NICK
	      on the new network. However, it is impossible to prevent this at
	      registration time. Thus, the IRC specs suggest to use the KILL command and
	      remove both the clients from the network. This is the approach used by
	      me in the project.

\end{itemize}

\subsection{PRIVMSG}

The PRIVMSG command works as follows: It takes the name of a target as the first parameter and the message
text as the final parameter. The server checks if the target is a user or
channel before making a decision on how to relay it forward. If the target is
found on the original server, the server can push the message to the target
user(s) message queues without relaying it to anyone. Otherwise, the server
must relay the message to send it to the destination client's server.

\begin{itemize}

	\item This command is only enabled after registration for both sender and
	      receiver.

	\item User can send messages to another user using their nick only.
	      This command can accept any nick that exists in the network as a valid
	      target.

	\item This command can also accept any channel name as a valid target. We only
	      support channels that are prefixed with \#. These characters have special meaning-
	      for example, they tell us if the channels are public or private.

	\item Since users can exist on different servers, the server has to relay
	      these messages to certain peers. Channel messages are different from user-user messages
	      because all members of a channel do not exist on the same server. The server does not know
	      where the members of the channel live, whereas a user can only live on one server. Therefore,
	      there are more messages relayed for a user-channel message as compared to a user-user message.

	\item If the message target is a user, the message is relayed to only one
	      peer in each step. When the final server connected to the target user receives the message,
	      it can deliver the message to the user and stop relaying. Every user-user
	      path is unique because the IRC network is a spanning tree
	      and does not contain cycles.

	\item If the message target is a channel, the message is broadcasted
	      to all of the peers of each server. This way, the message is propogated to every
	      other server on the network. we could decrease the number of messages sent if we knew where all the members of the channel lived.
	      At each step
	      of a channel relay message, the server does two things. First, it delivers the message
	      to any available channel members on the current server. Second, it relays the message to
	      its peers. The messages to the users are delivered using the queue system.
\end{itemize}

\section{Client}

The client implements a thread-based model with two threads:

\subsection{Main thread}

\begin{itemize}
	\item It is the duty of the main thread to read user input from stdin.
	\item The main thread blocks on the getline() instruction till the user types ENTER to send a line of text.
	\item If the input is a valid IRC
	      command, the string will be terminated by CRLF as required.
	\item If the input is a special command starting with a /, the client will
	      translate the string to the corresponding IRC command.
	\item The IRC command is added to the outbox queue of the user to send to the server through the worker thread.
	\item The main thread quits when the user enters the QUIT or SQUIT command.
\end{itemize}

\subsection{Worker thread}

\begin{itemize}
	\item This thread polls the server socket for read/write events.
	\item It displays the messages read from the server to stdout.
	\item It writes available messages from the outbox queue to the server.
	\item This thread quits when the server sends a ERROR message.
	      An ERROR message is also sent in response to the QUIT command. This allows the worker and main thread to exit gracefully.
\end{itemize}

The process to quit the program is as follows: The user types the
QUIT command. The main thread adds the final message to the queue and exits afterward. The worked thread sends
the QUIT message from the queue but waits for the response. The server replies to the QUIT command with a
ERROR message. The worker thread receives a ERROR message and exits.

The ERROR message is also sent in case of errors such as incorrect password. In this case,
the worker thread will exit while the main thread is stll alive. The main thread can now reconnect
to the server if required. As part of future work, the main thread could relaunch the worker after an error
has occured and prompt the user for the new registration message.

\section{Special commands}

The following commands have been designed for demonstration and are not part of
the specification.

The \verb|TEST_LIST_SERVER| command is asynchronous in nature, i.e. it involves communicating with the
entire network to create a response. Therefore, the reply is sent over a series of messages and we
maintain the current state of the response to decide when the response is completed.

The command \verb|TEST_LIST_SERVER| is used to get a list of all servers on the
network. Note that, in my implementation, each server knows which servers exist
in the network. However, I purposely implemented the command to watch my servers
communicate with each other to generate a response. To implement this command we use two kinds of
replies.

\begin{itemize}
	\item 901 \verb|RPL_TEST_LIST_SERVER|
	\item 902 \verb|RPL_TEST_LIST_SERVER_END|
\end{itemize}

The algorithm is described as follows:

\begin{itemize}

	\item A user requests the origin server for a list of servers on the network using the \verb|TEST_LIST_SERVER| command.
	\item The origin server adds the user to a map \verb|test_list_server_map| and initialises a struct
	      with the
	      following members:
	      \begin{itemize}
		      \item A pointer to the connection struct of the user.
		            This variable is used to send the response to the user in the future.
		      \item A set containing all the ``remaining" peers, filled with the current peers of the origin server:
		            this set inidicates which peers still need to send a reply. We can use this set to decide when the response is over.
	      \end{itemize}

	\item We initially send the client the names of the original server's peers. Each
	      server is returned in a new message of type \verb|RPL_TEST_LIST_SERVER|.
	\item Next, we make a \verb|TEST_LIST_SERVER| request to each of the peers in the set.
	      The client nick is used as an identifier in these requests to distinguish them.
	\item We can continue processing other commands from the user while waiting for
	      replies for
	      the \verb|TEST_LIST_SERVER| request.
	\item When one of the peers sends back a reply for the servers it has found, we relay the replies
	      the original client and remove the peer from the set. The peers perform this action recursively. In particular, when one of the peers sends a
	      \verb|RPL_TEST_LIST_SERVER| to the origin server, we add this response to the
	      original client's message queue, accessed by the connection struct.
	      The response contains the client nick which we can use
	      to index the \verb|test_list_server_map| for the data.
	\item When the set is empty, we have finished listing all
	      servers so we can send a \verb|RPL_TEST_LIST_SERVER_END| to the
	      original client and remove their entry from the map. If one of the servers leave,
	      they are removed from this set too, otherwise the client will wait forever.
	\item If during this period a new server joins the network, we
	      ignore them from the response. It is possible to add this peer to the set for
	      "real time" updates.
	\item If during this period the user sends another list command, we do not acknowledge it. This prevents us
	      from seeing duplicate server names in the response because the command is asynchronous.
\end{itemize}

\subsection{Network State}

In this section, I explain how I keep the state of the network consistent.

\begin{itemize}
	\item Event of server disconnect:
	      \begin{itemize}
		      \item Broadcast a SQUIT message for each server
		            behind that connection
		      \item Broadcast a QUIT for each client behind that
		            connection
	      \end{itemize}

	\item Event of new server connection:
	      \begin{itemize}
		      \item Authenticate server- both servers must share the passwords of the
		            other server in the PASS message and confirm that they match the configured
		            password. Both servers also share their name with each other with the
		            SERVER message.
		      \item Both servers share the names of the servers that they know about
		      \item Both servers share the nicks of the clients that they know of
		      \item Both servers share their channels with each other
		      \item At this stage, the servers also check for cycle detectioin and NICK collision
		            as explained before.
	      \end{itemize}

	\item Event of client disconnect:
	      \begin{itemize}
		      \item Remove client from all data structures
		      \item Broadcast a KILL for that client to all peers
	      \end{itemize}

\end{itemize}

\section{Routing messages}

Observe that the IRC network is a spanning tree structure, so it has no cycles.
We use a BFS like algorithm to route messages from one server to another.

Suppose we have three clients: Alice and Bob and Kate. There are three servers
A,B and C. There network graph has the following topology: $A \leftrightarrow B$ and $A
	\leftrightarrow C$. Further, suppose Alice is connected to server A, Bob is connected to server B and Kate is
connected to server C.

Suppose alice wants to send a message "hello" to kate.

The following actions will take place:

\begin{itemize}

	\item Alice will send a message \verb|PRIVMSG Kate :hello| to server A
	\item server A
	      will check if Kate is a client known to it.
	\item Since Kate is not a client on server
	      A, server A will relay this message to each of its peers in the
	      \verb|name_to_peer_map|, namely server B and server C.
	\item server A will add B
	      and C to a visited set to avoid relaying multiple times to a peer -
	      each peer can appear multiple times in the \verb|name_to_peer_map| because
	      there is a many-to-one correspondence between the servers and the peers.
	\item server B and server C take the same action when they recieve the message from A.
	\item Since Kate does not live on server B, server B
	      can ignore the message. B will not relay message back to A as
	      the messages are never relayed to the sender as it is redundant.
	\item server C will finally find that Kate lives on server C, so server C
	      will not relay this message any further.
	\item server C will add this message to Kate's message queue. When Kate is
	      ready to receive messages,
	      this server will send this message over a socket to Kate's client
	      program.
	\item the client program will display this message to Kate once it receives the
	      full message i.e. the CRLF bytes. This completes the journey of the message from client Alice
	      on server A to client Kate on server C.

\end{itemize}

As described above, the messages are propogated down the network until they reach
the destination or an edge node. The messages make their way to the destination
through a series of relays on intermediate nodes. This is the main idea of routing
messages in the IRC protocol. In fact, there are similarities between IRC and
distance vector routing, except there is no concept of shortest paths, as there is only one path.
This example was a general description of routing
messages, however, we can do better- Since each server knows where the client lives,
each server only needs to relay the message to one peer instead of all. As the servers
send advertisments, the other servers in the network can update their tables and learn
the "next hop" node for each target. If the server does not keep track of this data,
then it may relay everywhere and still have the same outcome.

\section{Conclusion}

I learnt a lot about network programming and distributed
systems from doing this project, and I also gained insight about the limitations
of the IRC protocol.

There are numerous ways to extend the functionality of the IRC
protocol and add support for features like file transfer or unicode emojis.
It is also interesting to consider more security features, as IRC
text-based authentication is not very strong.

In comparison to chat applications based on HTTP and WebSockets, the IRC
protocol has a smaller message size and no overhead of HTTP headers.
However, HTTP is a more flexible protocol, as we do not need
to know the position of parameters that appear in a message.
HTTP can be used to make more kinds of applications whereas IRC is specialised
only for texting-based applications. Moreover, IRC is a multipart protocol, as
longer messages need to be broken down into smaller ones and sent over a number
of replies.

Several limitations of the IRC protocol are mentioned in section 9.1 of the
RFC. There is an overhead of knowing the network state on every server. Also
it is hard to communicate all network updates with very little latency at the scale
of the IRC network. For such reasons, we should seek a more scalable protocol today.

\end{document}
